\section{Lecture 3 | Planning by Dynamic Programming}
The word dynamic programming can be split into two parts:
\begin{itemize} 
    \item Dynamic: sequential or temporal aspect of the problem
    \item Programming: optimisation of a program or a policy.
    \end{itemize}

The class of problems that dynammic programming can solve should have the following properties:
\begin{itemize}
    \item \textbf{Optimal substructure}: optimal solution can be decomposed into subproblems
    \item \textbf{Overlapping subproblems}: subproblems recur many times and 
    solutions can be cached and reused.
\end{itemize}
The framewwork that the RL is based on i.e. Markov Decision Process (MDP) has both 
of these properties. The Bellman equation provides the way to decompose the optimility problem
while the value function stores the solutions ans act as a cache to reuse the solutions.

In the dynamic programming, we assume that the MDP is known, and hence is used for the planning
in the MDP, and is not the solution to the complete RL problem with the machinery introduced
in the previous lectures.
Thus for prediction:
\begin{itemize}
    \item \textbf{Input}: MDP \(
        \langle \mathcal{S}, \mathcal{A}, \mathcal{P}, \mathcal{R}, \gamma \rangle
        \) and a policy \(\pi\) \emph{or} MRP \(
        \langle \mathcal{S}, \mathcal{P}^{\pi}, \mathcal{R}^{\pi}, \gamma \rangle
        \)
    \item \textbf{Output}: value function \(v_{\pi}\)
\end{itemize}
and for control:
\begin{itemize}
    \item \textbf{Input}: MDP \(
        \langle \mathcal{S}, \mathcal{A}, \mathcal{P}, \mathcal{R}, \gamma \rangle
        \)
    \item \textbf{Output}: optimal value function \(v_{*}\) and optimal policy \(\pi_{*}\)
    The optimal value function implies the optimal policy.
\end{itemize}
Apart from RL problems, dynamic programming can be used to solve other problems like
\begin{itemize}
    \item Schduling algorithms
    \item String algorithms
    \item Graph algorithms (e.g. shortest path algorithms)
    \item Bioinformatics
\end{itemize}

\subsection{Iterative Policy Evaluation}
The iterative policy evaluation is used to evaluate the given policy \(\pi\) in an MDP. 
Main idea to evaluate the value function \(v_{\pi}\) is to use the Bellman expectation equation,
and perform a full backup at each state. The full backup means that the value function is updated
using the value function of all the successor states. The update is done in an iterative manner
until the value function converges. The backup is a synchronous backup, i.e. the value function
is updated using the value function of the previous iteration for all the states. The algorithm
is given in the \autoref{alg:iterative_policy_evaluation}.


\begin{algorithm}[H]
    \caption{Iterative Policy Evaluation}
    \label{alg:iterative_policy_evaluation}
    \begin{algorithmic}[1]
        \State \textbf{Input:} Value function $v_{\pi}$
        \State \textbf{Initialize:} $v_{\pi}(s) \in \mathbb{R}$ arbitrarily for all $s \in \mathcal{S}$
        \While{$v_{\pi}$ not converged}
            \For{$s \in \mathcal{S}$}
                \State $v_{k+1}(s) \leftarrow \sum\limits_{a \in \mathcal{A}} 
                    \pi(a|s) \sum\limits_{s' \in \mathcal{S}} 
                    \left( 
                        R_{s}^{a} + \gamma \sum\limits_{s'' \in \mathcal{S}}
                        P_{ss'}^{a} v_{k}(s'')   
                     \right)$
            \EndFor
        \EndWhile
    \end{algorithmic}
\end{algorithm}

\subsection{Policy Iteration}
The policy iteration is used to find the optimal policy \(\pi_{*}\) in an MDP. The main idea
is to use the iterative policy evaluation to evaluate the policy \(\pi\) and then improve the policy
by acting greedily with respect to the value function \(v_{\pi}\). Thus creating a new policy
\(\pi^{\prime}  = \text{greedy}(v_{\pi})\), such that \(\pi^{\prime} \geq \pi  \).  The algorithm is given in the \autoref{alg:policy_iteration}.


\begin{algorithm}[H]
    \caption{Policy Iteration}
    \label{alg:policy_iteration}
    \begin{algorithmic}[1]
        \State \textbf{Input:} Value function $v_{\pi}$, Policy $\pi$
        \State \textbf{Initialize:} $v_{\pi}(s) \in \mathbb{R}$ arbitrarily for all $s \in \mathcal{S}$
        \While{$v_{\pi}$ not converged}
            \For{$s \in \mathcal{S}$}
                \State $v_{k+1}(s) \leftarrow \sum\limits_{a \in \mathcal{A}} 
                    \pi(a|s) \sum\limits_{s' \in \mathcal{S}} 
                    \left( 
                        R_{s}^{a} + \gamma \sum\limits_{s'' \in \mathcal{S}}
                        P_{ss'}^{a} v_{k}(s'')   
                     \right)$
                \State $\pi \leftarrow \text{greedy}(v_{\pi})$
            \EndFor
        \EndWhile
        \State \textbf{Output:} Optimal policy $\pi_{*}$
    \end{algorithmic}
\end{algorithm}

When the \autoref{alg:policy_iteration} converges, 
to the optimal value function \(v^{*}\), the policy \(\pi\) is the optimal policy \(\pi^{*}\).
Note that it is possible that the policy coverges to the optimal policy even if the value function
didn't converge to the optimal value function yet. Thus we can leverage this fact to stop the
policy iteration earlier if we only want the optimal policy and not the optimal value function.


The intution behind the idea can be explained as follows:

Consider a deterministic policy \(a = \pi(s)\). We can improve the policy by acting
greedily with respect to the value function
\[
        \pi^{\prime}(s) = \argmax_{a \in \mathcal{A}} q_{\pi}(s,a)
\]
This improves the value from any state \(s\) over one step of lookahead:
\[
    \begin{aligned}
            q_{\pi}(s, \pi^{\prime}(s)) 
            &= \max_{a \in \mathcal{A}} q_{\pi}(s,a) &&\dots\text{from the 
            defination of \(\pi ^{\prime}\)}\\
            &\geq q_{\pi}(s, \pi(s))\\
            &= v_{\pi}(s) &&\dots\text{from the defination of \(v_{\pi}\)}\\
    \end{aligned}
\]
It therefore improves the value function, by iterating this using a telescopic argument.
\[
        \begin{aligned}
            v_\pi (s) &\leq q_{\pi}(s, \pi^{\prime}(s)) \\
            &= \mathbb{E}_{\pi^{\prime}} \left[ R_{t+1} + \gamma v_{\pi}(S_{t+1}) |
             S_{t} = s \right] \\
                &\leq \mathbb{E}_{\pi^{\prime}} \left[ R_{t+1} + \gamma q_{\pi}(S_{t+1}, \pi^{\prime}(S_{t+1})) |
                S_{t} = s \right] \\
                &\leq \mathbb{E}_{\pi^{\prime}} \left[ R_{t+1} + \gamma R_{t+2} + \gamma^{2} v_{\pi}(S_{t+2}) |
                S_{t} = s \right] \\
                &\leq \mathbb{E}_{\pi^{\prime}} \left[ R_{t+1} + \gamma R_{t+2} + \dots |
                S_{t} = s \right] \\
                &= v_{\pi^{\prime}}(s) \\
        \end{aligned}  
\]
thus \(v_{\pi^{\prime}}(s) \geq v_{\pi}(s)\) for all \(s \in \mathcal{S}\). So we are
atleast not making things worse. Now, if the imporovemnts stop,
\[
    \begin{aligned}
        q_\pi (s, \pi^{\prime}(s)) &= \max_{a \in \mathcal{A}} q_{\pi}(s,a) \\
        &= q_{\pi}(s, \pi(s)) \\
        &= v_{\pi}(s) \\ 
    \end{aligned}
\]
which is basically the Bellman optimality equation. Thus we have found the optimal policy, 
since this policy is satisfying the Bellman optimality equation. Thus we can use the iterative
policy evaluation to evaluate the optimal policy.
\[
    \therefore v_\pi (s) = v_{\star} (s) \quad \forall s \in \mathcal{S}
\]

\subsection{Value Iteration}
Any optimal policy can be subdivided into two parts:
\begin{itemize}
    \item An optimal fisrt action \(A_{*}\)
    \item Followed by an optimal policy from the next state \(S^{\prime} \)
\end{itemize}

\begin{theorem}[Principal of Optimality]
    A policy \(\pi(a|s)\) achieves the optimal value from state \(s\) \(v_{\pi}(s) = v^{*}(s)\)
    if and only if for all \(s \in \mathcal{S}\) and \(a \in \mathcal{A}(s)\), \(\pi \) 
    achieves the optimal value from \(s^{\prime}\)
    \[
        v_{\pi}(s^{\prime}) = v^{*}(s^{\prime})
    \]
\end{theorem}

\subsubsection{Deterministic Value Iteration}
If we know the solution to subproblems \(v_{*}(s^{\prime})\) for all \(s^{\prime} \in \mathcal{S}\),
then we can easily solve the original problem by one step lookahead, and to apply this iteration
iterativly.
\[
    v_{\star}(s) \gets \max_{a \in \mathcal{A}} \left(   
    \mathcal{R} _{s}^{a} + \gamma 
    \sum\limits_{s' \in \mathcal{S}} \mathcal{P}_{ss'}^{a} v_{*}(s') \right)
\]
Thus, using the iterative application of the Bellaman optimality backup using synchronous backups
we get the \autoref{alg:deterministic_value_iteration}.
Unlike the policy iteration, the value iteration there is no explicit policy, 
and intermediate value functions may not correspond to any policy.
\[
    \implies v_2, v_3, \dots  \neq v_\pi \quad \text{for any policy} \pi  
\]


\begin{algorithm}[H]
    \caption{Deterministic Value Iteration}
    \label{alg:deterministic_value_iteration}
    \begin{algorithmic}[1]
        \State \textbf{Input:} Value function $v_{\pi}$
        \State \textbf{Initialize:} $v_{\pi}(s) \in \mathbb{R}$ arbitrarily for all $s \in \mathcal{S}$
        \While{$v_{\pi}$ not converged}
            \For{$s \in \mathcal{S}$}
                \State $v_{k+1}(s) \leftarrow \max\limits_{a \in \mathcal{A}} 
                    \left( 
                        R_{s}^{a} + \gamma \sum\limits_{s' \in \mathcal{S}}
                        P_{ss'}^{a} v_{k}(s')   
                     \right)$
            \EndFor
        \EndWhile
        \State \textbf{Output:} Optimal value function $v_{*}$
    \end{algorithmic}
\end{algorithm}

\subsubsection{Synchrounous Dynamic Programming Algorithms}
\begin{itemize}
    \item Algorithms are based on the state value function \(v_{\pi}\) or \(v_{\ast} (s)\)
    \item Complexity is \(\mathcal{O}(mn^{2})\) per iteration, for \(m\) actions and \(n\) states
    \item Could also apply to action value function \(q_{\pi}\) or \(q_{\ast}\)
    \item Complexity is \(\mathcal{O}(m^{2} n^{2})\) per iteration, for \(m\) actions and \(n\) states
    \end{itemize}
\begin{table}[H]
    \centering
    \begin{tabular}{c|c|c}
        \hline
         \textbf{Problem} & \textbf{Bellman Equation} & \textbf{Algorithm} \\
         \hline
         Prediction & Bellman Expectation Equation & Iterative Policy Evaluation \\
         \hline
         Control & Bellman Expectation Equation + Greedy Policy Improvement & Policy Iteration \\
         \hline
            Control & Bellman Optimality Equation & Value Iteration \\
         \hline
    \end{tabular}
    \caption{Synchronous Dynamic Programming Algorithms}
    \label{tab:synchronous_dynamic_programming_algorithms}
\end{table}

\subsubsection{Asynchronous Dynamic Programming Algorithms}

\subsubsection*{In-place Dynamic Programming}
The synchronous dynamic programming algorithms require the storage of two copies of the value
function, one for the current iteration and one for the next iteration.
\[
    \begin{aligned}
        v_{\text{new}}(s) &\gets \max_{a \in \mathcal{A}} \left(   
        \mathcal{R} _{s}^{a} + \gamma
        \sum\limits_{s' \in \mathcal{S}} \mathcal{P}_{ss'}^{a} v_{\text{old}}(s') \right) \\
        v_{\text{old}}(s) &\gets v_{\text{new}}(s) \quad \forall s \in \mathcal{S} \\
    \end{aligned}
\]
In-place dynamic programming is a way to save the storage by updating the value function inplace,
when it calculated.
\[
    v(s) \gets \max_{a \in \mathcal{A}} \left(
        \mathcal{R}_{s}^{a} + \gamma \sum\limits_{s' \in \mathcal{S}} \mathcal{P}_{ss'}^{a} v(s')
    \right) \quad \forall s \in \mathcal{S}
\]

\subsubsection*{Prioritized Sweeping}
The prioritized sweeping is a way to improve the efficiency of the dynamic programming algorithms,
by updating the states that are most likely to lead to large changes in other states, depending
on the magnitude of the Bellman error, where the Bellman error is defined as
\[
    \delta = \left \vert
        v(s) - \max_{a \in \mathcal{A}} \left(
        \mathcal{R}_{s}^{a} + \gamma \sum\limits_{s' \in \mathcal{S}} \mathcal{P}_{ss'}^{a} v(s')
    \right) \right \vert
\]
After the Bellman update, we upadte the Bellman error of the affeted states, thus requiring
the knowledge of the predecessors of the state. This method can be
implmented efficiently using a priority queue, where the priority of the state is the magnitude
of the Bellman error.

\subsubsection*{Real-time Dynamic Programming}
The real-time dynamic programming is a way to improve the efficiency of the dynamic programming
algorithms, by updating the states that are relevant to the agent, i.e. the states that the agent
is likely to visit. The agents's experience is used to guide the choice of the states to update.

\[
    v(S_t) \gets \max_{a \in \mathcal{A}} \left(
        \mathcal{R}_{S_t}^{a} + \gamma \sum\limits_{s' \in \mathcal{S}} 
        \mathcal{P}_{S_t s'}^{a} v(s')
    \right) \quad \forall s \in \mathcal{S}
\]

\subsubsection{Full-width Backups vs Sample Backups}
\begin{itemize}
    \item DP uses full width backups
    \item For ewach backup:
    \begin{itemize}
        \item Every successor state and action is considered
        \item Using the knowledge of the MDP transitions and reward function
    \end{itemize}
    \item DP is effective for small to medium sized MDPs
    \item For large MDPs, DP suffers from the Bellman's curse of dimensionality. i.e.
    the number of states grows exponentially with the number of state variables.
    \item Thus even one backup can be very expensive.
    \item Thus, we use sample backups, where we sample the successor state and action, using
    the sample rewards and transitions \(\langle S, A, R, S^{\prime} \rangle \)
    \item Advantage of sample backups:
    \begin{itemize}
        \item Model free: No knowledge of the MDP transitions is required
        \item Breaks the curse of dimensionality using sampling
        \item Cost of each backup is independent of the number of states
    \end{itemize}
\end{itemize} 